
%%
%% Custom Hyphenations
%%
%%
\hyphenation{cross-talk au-di-tory adap-t-a-tion phe-nom-en-o-lo-g-i-cal
  syn-a-pse co-inc-id-ence Tub-er-culo-vent-ral glyc-in-ergic psycho-phys-ical
  asym-met-ric ex-plor-at-ory pot-as-sium op-ti-mi-sa-tion au-di-t-ory sys-tem
  Neuro-in-for-mat-ics mar-g-in-al par-a-m-eters Rh-ode Neu-ro-fit-ter
  elec-tro-phys-io-log-i-cal Elec-tro-phys-io-log-i-cal prop-er-ties in-fer-ring the con-nec-tiv-i-ty with-in
  non-lin-ear ap-pr-oa-ch-es stel-late mi-cro-cir-cuit show-ing synap-tic
  in-ter-ac-tion iso-lam-inar pop-u-la-tion evo-l-ved com-part-ment
  con-duc-tance mod-els Hod-g-kin Hux-l-ey gluta-m-at-er-gic gen-o-mes
  Theu-nis-sen re-sp-on-se co-inc-id-ence det-e-ctor exp-eri-men-tal
  ac-cu-mu-lated neu-ro-science cre-ate de-tailed ex-ten-sively stud-ied
  In-tra-cel-lu-lar neur-rons in-tra-cel-lu-lar in-ves-ti-ga-tion se-quen-tial
  de-ter-min-ed cat-e-gor-is-ed im-me-di-ate sen-si-tiv-ity bicu-cu-line
  mul-ti-ple in-hib-it-ory com-mis-sural path-way re-cip-ro-cal fre-quency
  po-si-tion func-tion de-lay prep-a-ra-tions iso-lated oct-o-pus in-sen-si-tive
  Chop-per im-preg-na-tion phys-i-o-log-i-cal ef-fect GABA-er-gic in-puts on-set
  chop-pers pri-mar-ily gen-er-ate ar-ray gaus-sian ran-dom num-bers
  in-ves-ti-gat-ing im-p-or-tant phys-i-o-log-i-cal mech-a-nisms nox-ious
  stim-u-la-tion Neur-os-cience Re-search Com-mu-nic-a-tions De-part-ment
  Elec-trical Elec-tronic En-gin-eer-ing phe-nom-en-o-lo-g-i-cal-ly acous-tic
  sig-nal un-der-stood re-view Oh-lro-gge sim-u-la-tions per-formed us-ing
  neu-ral sim-u-la-tion pack-age con-ver-gence ven-tral  de-lay re-cep-tor
  dep-end-enc-ies int-er-dep-end-enc-ies}
%%
%% Sample custom-configuration
%%
%%   You are encouraged to modify the following section with any of your
%%   own custom commands, packages, etc.
%%

%error 'You should modify this section and remove this error.'

% for URLs
\usepackage{url}

% AMS packages
%\usepackage{amsfonts}
\usepackage{amssymb}
\usepackage[fleqn]{amsmath}   % displayed equations flush left
%\setlength{\mathindent}{0em}
%\usepackage{amsthm}
\usepackage[mathscr]{eucal}

\newcommand{\vect}[1]{\mathbf{#1}}


% Allow equations to break over pages...
\interdisplaylinepenalty=2500
% Command to stop equation breaks
% Note: enclose this in braces when used...
\newcommand{\donotsplitoverpages}{\interdisplaylinepenalty=10000}

%% Graphics
% \ifx\pdftexversion\undefined
%  \usepackage[dvips]{graphicx}
% \else
%  \usepackage[pdftex]{graphicx}
% \fi

%% My Graphics and Hyperlinks stuff
\usepackage{ifpdf}
 \ifpdf
   \pdfoutput=1
   \usepackage[pdftex]{graphicx}  % uncomment if using graphicx
   \usepackage[final,          % override "draft" which means "do nothing"
            colorlinks,     % rather than outlining them in boxes
            linkcolor=black, % override truly awful colour choices
            citecolor=black, %   (ditto)
            urlcolor=black,  %   (ditto)
            ]{hyperref}

\ifx\pdfoutput\undefined \usepackage[ps2pdf,
bookmarks=true,
bookmarksnumbered=true,
breaklinks=true,
final,          % override "draft" which means "do nothing"
colorlinks,     % rather than outlining them in boxes
linkcolor=black, % override truly awful colour choices
citecolor=black, %   (ditto)
urlcolor=black,  %   (ditto)
]{hyperref}

% \usepackage[pdftex]{hyperref}  % uncomment if using hyperref
%  \usepackage[ps2pdf]{thumbpdf}
 \DeclareGraphicsExtensions{.eps,.bmp}
 \else
 \DeclareGraphicsExtensions{.png,.pdf,.jpg,.JPEG}
 \usepackage{epstopdf}
 %\usepackage[pdftex,bookmarks=true,bookmarksnumbered=true,breaklinks=true]{hyperref}
  \pdfadjustspacing=1
  \usepackage[pdftex]{thumbpdf}
  \fi
  \else
   \usepackage[dvips]{graphicx}  % uncomment if using graphicx
    % comment if not using hyperref
 \usepackage[final,          % override "draft" which means "do nothing"
            colorlinks,     % rather than outlining them in boxes
            linkcolor=black, % override truly awful colour choices
            citecolor=black, %   (ditto)
            urlcolor=black,  %   (ditto)
             ]{hyperref}
 \DeclareGraphicsExtensions{.eps,.bmp}
\usepackage{breakurl} % necessary to break URLs when using LaTeX-> dvips -> Ps2PDF, must be after hyperref

\fi
% Enable IEEE macros
%\usepackage{IEEEtrantools}

% Use a plain bibliography style
%\bibliographystyle{plain}
% Use the IEEE bibliography style (sorted)
%\bibliographystyle{IEEEtrans}
% Use the IEEE bibliography style (unsorted; order of reference)
%\bibliographystyle{IEEEtran}

% For isolated bibliographies
\usepackage{bibunits}

\usepackage{color}
%\usepackage[noadjust]{cite}
\usepackage{caption}

% For cool tables
\usepackage{array}
\usepackage{tabularx}  % automatically adjusts column width in tables
\usepackage{multirow}  % allows entries spanning several rows
\usepackage{colortbl}  % allows coloring tables


% For algorithms
%\usepackage{algorithm}
%\usepackage{algorithmic}

% For cases
\usepackage{sublabel}

% For theroem numbers having the chapter included
%\usepackage{style/chngcntr}

% For cool theorem styles
%\usepackage[amsthm]{ntheorem}
%%\theorembodyfont{\normalfont}
%
%% Theorem definition
%\newtheorem{theorem}{Theorem}
%\counterwithin{theorem}{chapter}
%
%% Corollary definition
%\newtheorem{corollary}{Corollary}
%\counterwithin{corollary}{chapter}
%
%% Result definition
%\newtheorem{result}{Result}
%\counterwithin{result}{chapter}
%
%% Lemma definition
%\newtheorem{lemma}{Lemma}
%\counterwithin{lemma}{chapter}
%
%% Proposition definition
%\newtheorem{proposition}{Proposition}
%\counterwithin{proposition}{chapter}
%
%% Definition definition!
%\newtheorem{definition}{Definition}
%\counterwithin{definition}{chapter}
%
%% Remark definition (no counter?)
%\newenvironment{remark}{\emph{Remark:~}}{}
%
%% Fact definition (no counter?)
%\newenvironment{fact}{\emph{Fact:~}}{}

% (Re)Set the figure path
\newcommand{\setfigurepath}[1]{%
\ifx\figurepath\undefined
	\newcommand{\figurepath}{#1}
\else
	\renewcommand{\figurepath}{#1}
\fi%
}

% Used in the continued list environment below
\newcounter{continuedlist}

% Continued list environment
\newenvironment{continuedlist}{%
	\begin{enumerate}%
		% Space out each item
		\setlength{\itemsep}{1.25em}%
		% Start the enumeration from the previous value
		\setcounter{enumi}{\value{continuedlist}} %
}{ %
  % Save the counter to continue it later
  \setcounter{continuedlist}{\value{enumi}}%
  \end{enumerate}%
  % \vspace{1.25em}% 
  \vspace{1em}%
}

% Spaced out list environment
\newenvironment{spacedoutlist}{%
	\begin{itemize}%
		% Space out each item
		\setlength{\itemsep}{1.25em}%
}{\end{itemize}}


%% My added packages


\usepackage[usenames,dvipsnames]{xcolor}
\definecolor{halfgray}{gray}{0.55}

\usepackage{listings}
\lstset{language=C++,%[LaTeX]Tex,%
    keywordstyle=\color{RoyalBlue},%\bfseries,
    basicstyle=\small\sffamily,
    identifierstyle=\color{NavyBlue},
    commentstyle=\color{Green}\rmfamily,
    stringstyle=\sffamily,
    numbers=left,%none,%
    numberstyle=\scriptsize,%\tiny
    stepnumber=5,
    numbersep=8pt,
    showstringspaces=false,
    breaklines=true,
    %frameround=ftff,
    %frame=single
    %frame=L
    lineskip=-5pt
}


% \lstset{language=Octave,                % choose the language of the code
% basicstyle=\footnotesize,       % the size of the fonts that are used for the code
% numbers=left,                   % where to put the line-numbers
% numberstyle=\footnotesize,      % the size of the fonts that are used for the line-numbers
% stepnumber=2,                   % the step between two line-numbers. If it's 1 each line will be numbered
% numbersep=5pt,                  % how far the line-numbers are from the code
% backgroundcolor=\color{white},  % choose the background color. You must add \usepackage{color}
% showspaces=false,               % show spaces adding particular underscores
% showstringspaces=false,         % underline spaces within strings
% showtabs=false,                 % show tabs within strings adding particular underscores
% frame=single,			% adds a frame around the code
% tabsize=2,			% sets default tabsize to 2 spaces
% captionpos=b,			% sets the caption-position to bottom
% breaklines=true,		% sets automatic line breaking
% breakatwhitespace=false,	% sets if automatic breaks should only happen at whitespace
% lineskip=-5pt  %
% }

\ifx\setcitestyle\undefined
\usepackage[sort,round,authoryear]{natbib}
\setcitestyle{aysep={}} % J Neurophys formatting
\fi

\usepackage{xspace}
\usepackage{rotating}
\usepackage{tikz}
\usepackage{calc}

\newcommand{\hdr}[3]{%
\multicolumn{#1}{|l|}{\color{white}\cellcolor[gray]{0.0}%
\textbf{\makebox[0.05\linewidth][l]{#2}\hspace{0.45\linewidth}\makebox[0pt][c]{#3}}%
%\textbf{\makebox[0pt]{#2}\hspace{0.5\linewidth}\makebox[0pt][c]{#3}}%
}}
 % Nordelie table environment
\newenvironment{ntab}[4]{
\noindent\begin{tabularx}{\linewidth}{#1}\hline 
\multicolumn{#2}{|l|}{\color{white}\cellcolor[gray]{0.0}\textbf{#3\hfill{}{#4}\hfill{}}}
}{
\end{tabularx}
\vspace{1ex}
}



\usepackage[colorinlistoftodos,backgroundcolor=yellow!35,textsize=footnotesize]{todonotes}
%todonotes depends on ifthen, tikz, calc, xkeyval, graphicx (is loaded via the
%tikz package), and xcolor

\setlength{\marginparwidth}{2cm}
\newcommand{\yellownote}[1]{\todo[inline]{#1}}
%\newcommand{\yellownote}[1]{\todo{#1}}
\newcounter{mycomment}
\newcommand{\mycomment}[2][]{%
% initials of the author (optional) + note in the margin
\refstepcounter{mycomment}%
{%
\setstretch{0.7}% spacing
\todo[color={red!100!green!33},size=\small]{%
\textbf{Comment [\uppercase{#1}\themycomment]:}~#2}%
}}


\usepackage{scrtime}
%\usepackage{mparhack}

%\setlength{\parskip}{0ex} or {1.2ex}
%\setlength{\parindent}{0em} or {0ex}

\usepackage[australian]{babel}

\usepackage{booktabs,ltxtable,ctable,dcolumn}
\newcommand{\otoprule}{\midrule[\heavyrulewidth]}

\usepackage{doipubmed}

% For subfigures
\usepackage{keyval}
\usepackage[config,labelfont={sf,bf}]{subfig}
%\captionsetup[table]{position=top}
%\captionsetup[subtable]{position=top}
%\usepackage[heightadjust=all,valign=t]{floatrow}
%\usepackage{fr-subfig}
%\floatsetup{style=Plaintop}
%\usepackage{subfigure}

\usepackage{lscape}

\newcommand{\code}[1]{\mbox{\normalfont\texttt{#1}}}
\newcommand{\progname}[1]{\mbox{\normalfont\textsf{#1}}}
\newcommand{\figfont}[1]{\large{\textbf{\textsf{#1}}}}
